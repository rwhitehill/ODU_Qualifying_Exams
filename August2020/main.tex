\section{August 2020}

\subsection*{Classical Mechanics}
\addcontentsline{toc}{subsection}{Classical Mechanics}

\prob{1.1}{

Consider a simple (plane) pendulum consisting of a mass $m$ attached to a massless rod of length $l$.
After the pendulum is set into motion (at $t = 0$), the length of the rod is lengthened at a constant rate ${\rm d}l/{\rm d}t = u$.
The suspension point remains fixed.

\begin{parts}
    \item Write down the Lagrangian for the pendulum.

    \item Obtain the equations of motion, but do not solve them.
    Show that they reduce to the equation of motion for a fixed-length pendulum when $u = 0$.

    \item Obtain the Hamiltonian of the system.

    \item Calculate the total mechanical energy of the system and compare it to the Hamiltonian.

    \item The energy of the system is not conserved.
    What is the rate of change of the energy?
\end{parts}

}

\sol{}


\prob{1.2}{

A relativisitic particle of rest energy $mc^2$ ($c$ is the speed of light) and charge $q$ is constrained to move in the $xy$-plane, and is under the influence of a uniform and constant electric field $\vb*{E} = E \vu*{x}$ directed along the $x$-axis.
At time $t = 0$, the particle has position $\vb*{r}_0 = (0,0)$ and momentum $\vb*{p}_0 = (0,p_0)$.

\begin{parts}
    \item Write down and solve the equations of motion in the $x$- and $y$-directions.

    \item Obtain the particle's trajectory in the $xy$-plane, i.e. $x = x(y)$.

    \item Obtain the trajector in the non-relativistic case, in which the particle's speed is much smaller than the speed of light.
\end{parts}

}

\sol{}


\prob{1.3}{

Find the trajectory of a particle in the field
\begin{align*}
    U(r) = -\frac{\alpha}{r} + \frac{\beta}{r^2}
,\end{align*}
with $\alpha > 0$ and $\beta > 0$.
Under what condition is the trajectory closed?

\textbf{Hint}: The following indefinite integral may be useful
\begin{align*}
    \int \frac{\dd{x}}{\sqrt{1 - x^2}} = -\arccos{x}
.\end{align*}

}

\sol{}


\prob{1.4}{

Four massless rods of length $L$ are hinged together at their ends to form a rhombus.
A particle of mass $m$ is attached at each joint.
The opposite corners of the rhombus are joined by springs, each with spring constant $k$.
In the equilibrium square configuration, the springs are unstretched.
The motion is confined to a plane, and the particles move only along the diagonals of the rhombus.

\begin{parts}
    \item The system has a single degree of freedom.
    Provide an explanation for why this is so.

    \item Choose a suitable generalized coordinate and obtain the Lagrangian.

    \item Deduce the equation of motion and obtain the frequency of small oscillations about the equilibrium configuration.
\end{parts}

}

\sol{}


\prob{2.1}{

A particle moves in one dimension under the influence of a potential $V(x) = F |x|$, where $F$ is a constant.
Using the action-angle variables, find the period of the motion as a function of the particle's energy.

}

\sol{}


\subsection*{Electricity \& Magnetism}
\addcontentsline{toc}{subsection}{Electricity \& Magnetism}

\prob{2.2}{

An infinite solenoid with $N$ coils per unit length and a cross-sectional area $\pi a^2$ carries a current $I_0$.
A circular plastic hoop of radius $b > a$ surrounds the solenoid and is oriented in such a way that the plane of the hoop is perpendicular to the solenoid axis.
A particle of charge $q$ and mass $m$, initially at rest, is constrained to slide freely (without friction) along the hoop.
if the current in the solenoid is turned off slowly, find the velocity of the particle after the current has reached zero.
ignore the effects of radiation or gravity.

}

\sol{}


\prob{2.3}{

A grounded conducting sphere of radius $a$ is concentric with an insulated spherical shell of radius $b$ ($b > a$).
A charge with surface density $\sigma_0 \cos{\theta}$ is distributed over the spherical shell.
Find the surface charge density on the sphere.

}

\sol{}


\prob{2.4}{

A particle of charge $q$ and mass $m$ moves in a uniform and constant magnetic field $\vb*{B} = B \vu*{e}_z$, and has velocity $\vb*{v}_0 = v_0 \vu*{e}_x$ ($v_0 \ll c$, where $c$ is the speed of light) at $t = 0$.
How long wil it take for the particle to lose half of its kinetic energy to radiation?
(Assume that the magnetic field is sufficiently weak so that the particle loses half of its energy after many revolutions).

}

\sol{}


\prob{3.1}{

An electric dipole $\vb*{p}$ is oriented perpendicularly to the surface of conducting sphere of radius $a$, and is located a distance $d > a$ away from its center.
The sphere is grounded.
Calculate the location, strength, and direction of the image dipole.

}

\sol{}


\prob{3.2}{

The mean lifetime of the $K_L^0$ ($K$-long) particle is $\tau \approx 5 \times 10^{-8}~{\rm s}$.
A beam of $K_L^0$'s with momenta $P = 5 ~{\rm GeV}/c$ is produced at a distance of $25~{\rm m}$ from the target of the experiment.

\begin{parts}
    \item Find the fraction of the beam particles that would have decayed by that time.

    \item What fraction of the beam particles would have decayed if their momenta were $P = 0.3 ~{\rm GeV}/c$.
\end{parts}

}

\sol{}


\prob{3.3}{

Consider a beam of protons of density $n$ (i.e., $n$ is the number of protons per unit volume), velocity $\vb*{v}$, and cross sectional area $S$.

\begin{parts}
    \item Calculate the current and charge density $\rho$ in the laboratory frame $K$, and the charge density $\rho_0$ in the rest frame $K_0$ of the protons.

    \item An electron is moving at a distance $d$ from the beam with velocity $-\vb*{v}$ in $K$ (i.e., the electron velocity is equal in magnitude to that of the protons, but in the opposite direction).
    Calculate the force acting on the electron in $K$ by first calculating it in the rest frame $K'$ of the electron and then transforming it back to $K$.

    \item Assuming $d > \sqrt{S/\pi^2}$, calculate the electric and magnetic fields generated by the beam of protons in $K$, and obtain the force acting on the electron in this frame.
    Does it agree with the force obtained in part (b) above?
\end{parts}

\textbf{Hint}: Denote as $\vb*{F}_{\parallel}$ the component of a force along the velocity $\vb*{v}$ between two frames $K$ and $K'$, and as $\vb*{F}_{\perp}$ the component transverse to it.
Then these components in the two frames are related in the following way: $\vb*{F}_{\parallel} = \vb*{F}_{\parallel}$ and $F_{\perp} = \vb*{F}_{\perp}'/\gamma$, where $\gamma = 1/\sqrt{1 - v^2/c^2}$.

}

\sol{}


\subsection*{Quantum Mechanics}
\addcontentsline{toc}{subsection}{Quantum Mechanics}


\prob{3.4}{

Consider a system with a pair of observables $A$ and $B$, whose commutation relations with the Hamiltonian $H$ take the form $[H,A] = i w B$ and $[H,B] = -iwA$, where $w$ is some real constant.
Assume that the expectation values of $A$ and $B$ are known at time $t = 0$.
Give formulas for the expectation values of $A$ and $B$ as a function of time.

}

\sol{}


\prob{4.1}{

At time $t = 0$, the state of a free one-dimensional particle is described by the wave function
\begin{align*}
    \Psi(x,t=0) = A \exp( -\frac{x^2}{2a^2} + i \frac{m v_0 x}{\hbar} )
,\end{align*}
where $|A|^2 = \frac{1}{\sqrt{\pi a^2}}$.

\begin{parts}
    \item Find the wave function at arbitrary time $t$.

    \item Find the averages of position and momentum, respectively $\overline{x}(t)$ and $\overline{p}(t)$.
\end{parts}

\textbf{Hint}: You may need standard Gaussian integrals
\begin{align*}
    \int_{-\infty}^{\infty} \dd{x} e^{-(ax^2 + bx)} = \sqrt{\frac{\pi}{a}} e^{b^2/4a} \qquad a > 0
,\end{align*}
and
\begin{align*}
    \int_{-\infty}^{\infty} \dd{x} e^{i(ax^2 + bx)} = \sqrt{\frac{i \pi}{a}} e^{-ib^2/4a} \qquad a > 0
.\end{align*}

}

\sol{}


\prob{4.2}{

A particle with spin-1/2 and mass $M$ is confined to move along a thin ring of radius $R$ in the $xy$-plane.
The Hamiltonian is given by
\begin{align*}
    H = H_0 + \frac{2 \alpha}{\hbar^2} \vb*{L} \cdot \vb*{S} + \frac{2 \beta}{\hbar} S_x
,\end{align*}
where $H_0$ is the Hamiltonian for the free orbital motion of the particle along the ring, $\vb*{L}$ is the orbital angular momentum operator, $\vb*{S}$ is the spin operator with $\vb*{S} = (\hbar / 2)(\sigma_x,\sigma_y,\sigma_z)$, and the $\sigma_i$'s are the Pauli matrices.
The second term in $H$ describes the spin-orbit coupling, while the third term represents the Zeeman energy associated with the magnetic field applied along the $x$-axis.

Calculate the energy levels of the system.
Are the energy levels degenerate?

}

\sol{}


\prob{4.3}{

The Hamiltonian describing the interaction of the electron spin with a magnetic field $\vb*{B}$ is given by
\begin{align*}
    H = - \frac{e}{mc} \vb*{S} \cdot \vb*{B}
,\end{align*}
where $c$ is the speed of light, and $e$ and $m$ are the charge and mass of the electron, respectively.
The spin operator $\vb*{S}$ is given by $\vb*{S} = (\hbar / 2) (\sigma_x,\sigma_y,\sigma_z)$, where the $\sigma_i$'s are the Pauli matrices.

\begin{parts}
    \item Find that the Heisenberg equation of motion for
    \begin{align*}
        \vb*{S}(t) = e^{i H t / \hbar} \vb*{S} e^{-i H t / \hbar}
    .\end{align*}

    \item Solve the equation of motion for $\vb*{S}$ when $\vb*{B} = (0,0,B)$.
\end{parts}

}

\sol{}


\prob{4.4}{

Consider a beam of spin-1/2 neutral particles that is perpendicularly incident on a block of ferromagnetic material.
Let the direction of the incident beam be in the $x$-direction, and let the surface of the ferromagnetic material be in the $yz$-plane.
The ferromagnetic material fills the entire $x > 0$ region.
The incident particles all have energy $E$ and mass $m$, and a magnetic moment $\vb*{\mu} = \gamma \vb*{s}$ with $\gamma < 0$.
They are subject to a potential energy consisting of two terms.
The first one corresponds to the interaction of the particle with the substance, and is simply represented by $V(x) = 0$ for $x \leq 0$, and $V(x) = V_0 > 0$ for $x > 0$.
The second term corresponds to the interaction of the magnetic moment with the internal magnetic field $\vb*{B}_0$ of the material.
The field $\vb*{B}_0$ is assumed to be uniform in the $z$-direction.
Thus the potential associated with this interaction is given by $W(x) = 0$ for $x \leq 0$, and $W(x) = \omega_0 s_z$ for $x > 0$ with $\omega_0 = -\gamma B_0$.
Assume that
\begin{align*}
    0 \leq \hbar \omega_0 < V_0, \quad V_0 - \hbar \omega_0/2 < E < V_0 + \hbar \omega_0 / 2
.\end{align*}

\begin{parts}
    \item Determine the eigenfunctions of the particle (of energy $E$ in the range above) which correspond to a positive incident momentum along the $x$-axis and spin either parallel or antiparallel to the $z$-axis.
    In particular, calculate the transmission coefficient for these two cases.

    \item Assuming that the incident beam is unpolarized, calculate the polarization of the reflected beam.
\end{parts}

\textbf{Hint}: Let $N$ be the number of particles in the incident beam.
The number of reflected particles with spin up or down are, respectively, $(N/2) R_+$ or $(N/2) R_-$, where $R_+$ and $R_-$ are the reflection coefficients.

}
